\chapter{Pre-Lab 6\&5}
\setcounter{TASignatures}{0}
\setcounter{AsideCounter}{0}

\section{Introduction}

Make sure to complete this pre-lab before your assigned lab time. You will not be allowed to begin working on your lab without having this complete.

\section{Background}


In the coming lab you will continue to use timer instructions, math instructions, and boolean logic. As an added component, you will also be making the HMI display screen for each of the portions of the lab. 

The coming lab will be extended across two lab times. So, you will not have a pre-lab next week!

\section{Problem 1}
Describe in detail the functionality for each of these HMI objects:
\begin{enumerate}
    \item Momentary Push Button
    \item Numeric Entry
    \item Numeric Display
    \item Text Box
    \item Multi-State Indicator
\end{enumerate}


\section{Problem 2}

Draw an HMI interface and state all necessary PLC tags required to set the speed of a centrifuge and display the current speed of the centrifuge. The operator should be able to enter a speed and then press a button to confirm that they want to set the centrifuge to the entered speed.

**If you are confused or feel that some information has not been given which is necessary, make assumptions to fill in the gaps and clearly state the assumptions that you have made (if any).

\section{Problem 3 - Read the Manual}

Read the lab manual. Then write a paragraph about the content and expectations in the lab manual which will convince the grader that you have in fact read the complete lab manual.