\chapter{Pre-Lab 8\&9}
\setcounter{TASignatures}{0}
\setcounter{AsideCounter}{0}

\section{Introduction}

Make sure to complete this pre-lab before your assigned lab time. You will not be allowed to begin working on your lab without having this complete.

\section{Background}


In the coming lab you will continue to use timer instructions, math instructions, and boolean logic. As an added component, you will also be making the HMI display screen for each of the portions of the lab. 


\section{Problem 1}
Read the lab 8\&9 manual and familiarize yourself with the program specifications. 

In the hints section that follows the program specifications there is a suggested approach to programming this lab. Follow the approach listed there, writing each of the specifications and the steps that you think will be involved. Then write the ladder logic (on paper) that you think will allow you to keep track of the average speed without keeping a history of past speeds. 


\section{Problem 2}

Draw an HMI interface and state all necessary PLC tags required to realize the HMI specifications laid out in the Lab 9 manual.

**If you are confused or feel that some information has not been given which is necessary, make assumptions to fill in the gaps and clearly state the assumptions that you have made (if any).

\section{Problem 3 - Read the Manual}

Read the lab manual. Then write a paragraph about the content and expectations in the lab manual which will convince the grader that you have in fact read the complete lab manual.