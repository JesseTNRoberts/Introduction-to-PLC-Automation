\chapter{Pre-Lab 3}
\setcounter{TASignatures}{0}
\setcounter{AsideCounter}{0}

\section{Introduction}

Make sure to complete this pre-lab before your assigned lab time. You will not be allowed to begin working on your lab without having this complete.

\section{Background}


In the coming lab we will be using math and compare instructions. However, there are some instructions that are out of bounds. Specifically, those listed at the beginning of the Lab 3 manual. Don't use any of the prohibited instructions in this pre-lab either.


\section{Problem 1}

The first problem is to calculate the modulo of two operands without using the modulo command. To get more information regarding the modulo operation in general, refer to wikipedia. 

Write a ladder logic program on paper to calculate the modulo operation (written as \%) of \verb|OperandA| and \verb|OperandB|. You are only allowed to use instructions for truncate, multiply, divide, add, and subtract. You are obviously not allowed to use the mod instruction. Store the result in \verb|Calculated_Modulo|.


\aside{The TRN instruction can be found in the Allen Bradley instruction set manual that is available. The truncate instruction removes the decimal portion of a number and leaves it whole without any rounding. ie. $TRN(6.1)=6.0$ and $TRN(6.99)=6.0$}


\section{Problem 2}

Write a ladder logic program on paper to store x and y coordinates for a given \verb|Angle| on the unit circle in \verb|x| and \verb|y| respectively.

Hint: The unit circle has a radius of 1. Use this information and the sin/cos instructions to accomplish this.


\section{Problem 3 - Read the Manual}

Read the lab manual. Then write a paragraph about the content and expectations in the lab manual which will convince the grader that you have in fact read the complete lab manual.